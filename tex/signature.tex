%!TEX root = ../main.tex

\section{Digital Signatures}
A short recap of digital signatures.

A signature scheme comprises key generation, signing and verification functions.
Key generation creates a public and secret (private) key pair ($pk, sk$).

Signing function takes an arbitrary message $m$ and the secret key $sk$ and outputs a signature $\sigma$.
\[
\sigma\leftarrow Sign(m,sk)
\]

A verify function takes a message $m$, signature $\sigma$ and the public key $pk$. The verification function returns $\true$, if $\sigma$ was produced for $m$ with the secret key $sk$ matching $pk$. Otherwise verification returns \false.
\[
\{ \true, \false \} \leftarrow Verify(m,\sigma,pk)
\]

Bitcoin currently uses ECDSA signatures, i.e. signatures where the public key is a point on an elliptic curve.

\section{Account balances and UTXO}
The most prominent use case for blockchains is currently digital currency. 
We now spend some time to look at how to implement money transfer. 

\question{How would you implement a digital bank?}

\begin{idea}
	\label{idea-pk}
	We can use public keys as identity for users or owners of money. 
\end{idea}
\begin{note}
Idea~\ref{idea-pk} allows to avoid the use of \emph{public key infrastructure} (PKI), that is otherwise used to bind public keys to identities.
\end{note}

\textbf{Algorithm~\ref{alg:bank}} show a simple bank that maintains balances.
The algorithm has two transactions, \textsc{Create} creates new balances.
\textsc{Transfer} allows to transfer money from one account to another.
The algorithm assumes that balances that have not been initialized have the value 0.


\textbf{Authentication} of transactions poses a problem:
\begin{itemize}
	\item For \textsc{Create} transactions we assume that they are are only valid during setup of the system.
	\item For \textsc{Transfer} transactions it is desired that all transactions can be submitted by users. E.g. if balances are identified with a public key, we could require transactions to be signed by the owner of \textit{pk-from}.
\end{itemize}

\noindent
A bank based on Algorithm~\ref{alg:bank} may be susceptible to \textbf{replay attacks}. A \textsc{Transfer} transaction signed by \textit{pk-from} may be submitted multiple times, given that the account of \textit{pk-from} has sufficient fonds. This will result in additional fonds transferred to \textit{pk-to}.

\begin{algorithm}[ht!]
	\caption{Simple Bank using account balances}
	\label{alg:bank}
	\begin{algorithmic}
		\State{$balances :=[pk]\textnormal{uint}$}
		\Procedure{Create}{$value, pk$}
			
				\State{$balances[pk] += value$}
			
		\EndProcedure
		\Procedure{Transfer}{$value, \textit{pk-from}, \textit{pk-to}$}
			\If{$balances[\textit{pk-from}] > value$ }
				\State{$balances[\textit{pk-from}] -= value$}
				\State{$balances[\textit{pk-to}] += value$}
			\EndIf
		\EndProcedure
	\end{algorithmic}
\end{algorithm}

\question{How could you implement this bank using a blockchain?}

\begin{note}
In a centralized system, a trusted party could process transactions, compute balances and distribute balances to all parties. 

Without a trusted party, it is necessary that transactions are distributed to everybody. Every party can then deterministically compute the balances.

One problem in this model is how to role back transactions.
\end{note}

\subsection{UTXO}
We now explain the \emph{unspent transaction output} (UTXO) model. 

\begin{idea}[UTXO]
The main idea behind the \emph{UTXO} model is that, given a record of transactions, validating that a certain balance is greater than $value$ is the same as verifying that this account has received $value$ funds that have not been spend yet.

Note that it is usually not necessary to look at all transactions received by the sender, but only a subset. The \emph{UTXO} model allows the transaction to specify the subset we should look at.
\end{idea}

\begin{definition}
	A \textbf{transaction output} is a tuple $(v, pk)$ that shows, that $v$ funds have been transferred. $pk$ is a \emph{spending condition} that must be met to spend claim $v$. Typically $pk$ requires a signature with a given public key.
	
	A \textbf{transaction input} is a tuple consisting of a reference to a transaction output and an argument that meets the outputs condition.
	I.e. $(outp_i,\sigma)$ where $\sigma$ is \emph{redeeming argument} a matching $pk$, e.g. a signature.
	
	A \textbf{transaction} is a tuple containing a list of transaction inputs and a list of new outputs.
\end{definition}


\begin{note}
	\label{bitcoin:transactions}
	In bitcoin transactions are implemented in the following way:
	\begin{itemize}
		\item An output from transaction $t$ is identified by a tuple $(h_t,i)$,
		where $h_t$ is the hash of $t$ and $i$ is the index in the list of outputs in $t$.
		\item Algorithm~\ref{alg:transact} shows how a transaction is validated.
		For a transaction to be valid, \emph{all inputs must be unspent}, input \emph{signatures must validate} and the \emph{sum of input values must be larger than the  output values}.
		\item Algorithm~\ref{alg:transact} ensures that a transaction can only be validated once and no two valid transactions can spend the same output.
		\item The different between transaction inputs and outputs is called transaction \emph{fee}.
		\item Example~\ref{ex:P2PKH} gives an example for more complex conditions that may be required to spend an output.
		\item When the value of inputs is larger than the desired value to be spend, it is common to create an additional output that contains change.
	\end{itemize}
	
\end{note}

\begin{algorithm}[h!]
	\caption{Transaction validation and maintenance of UTXO}
	\label{alg:transact}
	\begin{algorithmic}
		\State{$UTXO := map[(h,i)]\rightarrow(value, pk)$}
		\Procedure{transfer}{$inputs, outputs$}\Comment{Transaction $t$ with hash $h_t$}
			\For{$((h,i),\sigma)\in inputs$}
				\If{$UTXO[(h,i)]$ does not exist}
					\State{\textbf{abort}}
					\Comment{invalid transaction}
				\EndIf
				\If{$\text{verify}(h_t,\sigma, UTXO[(h,i)].pk) == \false$}
					\State{\textbf{abort}}
					\Comment{invalid transaction}
				\EndIf
			\EndFor
			\If{ sum of values of inputs $<$ sum of values of new outputs}
				\State{\textbf{abort}}
				\Comment{invalid transaction}
			\EndIf
			
			\For{$((h,i),\sigma)\in inputs$}
				\State{$UTXO[(h,i)]=nil$} \Comment{output spent}
			\EndFor
			\State{$h_t:= hash(transaction)$}
			\State{$UTXO[h_t]=outputs$}
			\Comment{add new output}
			
			
		\EndProcedure
	\end{algorithmic}
\end{algorithm}


\begin{definition}
A \emph{double-spend} is a situation where multiple transactions attempt to spend the same output. 	
\end{definition}
\begin{note}
Note that according to Algorithm~\ref{alg:transact} only one of double-spend transactions can be validated. 
\end{note}

\pagebreak


\begin{example}
	\label{ex:P2PKH}
	The most common examples for arguments necessary to claim a transaction output are listed here. In Bitcoin they are expressed in a stack based scripting language.
	\begin{enumerate}[label=\alph*)]
		\item A signature that matches a certain public key.
		\item A public key that hashes to a certain value and a signature that matches this key. (Pay to public key hash or P2PKH).
		\item Multiple signatures that match a sequence of public keys. (Multisig)
		\item Multiple signatures that match $m$ out of $n$ provided public keys.
		\item A script that hashes to a certain value and an argument that causes this script to evaluate to true. (pay to script hash)
	\end{enumerate}
	See \href{https://github.com/bitcoinbook/bitcoinbook/blob/develop/ch06.asciidoc#P2PubKHash1}{here} and \href{https://github.com/bitcoinbook/bitcoinbook/blob/develop/ch06.asciidoc#scriptSig_and_scriptPubKey}{here} for explanation of P2PKH script.
\end{example}

\begin{note}
To maintain a copy of our transaction based bank, a node has to maintain the set of all unspent transaction outputs $UTXO$. 

If variant b) is used instead of variant a) from Example~\ref{ex:P2PKH} this may significantly reduce the size of the $UTXO$ data structure. The same holds, if d) is used instead of c).
\end{note}	

\comment{On blackboard, give example of P2PKH script.}

\begin{definition} An \textbf{address} is either a public key or the hash of a public key. Given the address of a user, it is possible to transfer funds to this user, i.e. create an output that this user can claim by providing a correct signature.
\end{definition}

\begin{note}
Bitcoin and many other cryptocurrencies use Base-58 encoding. This encoding uses small and large letters (a to z) and numbers, omitting 0 (number zero), O (capital o), l (lower L) and I (capital i) because of their ambiguity.

Bitcoin addresses use Base58Check encoding adds a 4 byte checksum before Base-58 encoding, to protect against typos, ...
\end{note}


\subsection{Privacy in the UTXO model}
\comment{Not covered in lecture}
Different from the account and balance system, the $UTXO$ model encourages the use of different addresses. This makes it harder to identify all transactions done by a single user.

However research has shown, that based on transactions, it is easy to identify different addresses belonging to the same user.

On the other hand, $UTXO$ allows to trace in which transactions a particular value or coin was involved.

\begin{definition}
A \textbf{tainted coin} is a transaction output that is either the result of a transaction considered unethical or illegal or derived from the output of such a transaction by multiple other transactions.
\end{definition}

\begin{note}
Based on the concept of tainted coins it is debatable whether digital cash based on the UTXO model is \emph{fungible}. In economics fungibility is defined as the property that any two units of a good are interchangeable.
\end{note}

\question{What about paper money? Is it fungible?}

The UTXO model allows to create a mixing service:
\begin{definition}
	A \emph{mixing service} can be used to prevent a third party from tracking a specific users transactions. A mixing service would receive payments from many users, and pay them back using new addresses. 
\end{definition}

\begin{note}
	\begin{itemize}
		\item A mixing service makes it hard to see which of the new addresses belongs to which of the users that sent money to the mixing service.
		\item A mixing service usually requires a high fee.
		\item A mixing service is usually implemented as a centralized entity.
	\end{itemize}
\end{note}