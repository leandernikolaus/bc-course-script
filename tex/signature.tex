%!TEX root = ../main.tex

\section{Digital Signatures}
A short recap of digital signatures.

A signature scheme comprises key generation, signing and verification functions.
Key generation creates a public and secret (private) key pair ($pk, sk$).

Signing function takes an arbitrary message $m$ and the secret key $sk$ and outputs a signature $\sigma$.
\[
\sigma\leftarrow Sign(m,sk)
\]

A verify function takes a message $m$, signature $\sigma$ and the public key $pk$. The verification function returns $\true$, if $\sigma$ was produced for $m$ with the secret key $sk$ matching $pk$. Otherwise verification returns \false.
\[
\{ \true, \false \} \leftarrow Verify(m,\sigma,pk)
\]

Bitcoin currently uses ECDSA signatures, i.e. signatures where the public key is a point on an elliptic curve.

\begin{example}
One example for digital signatures is https certificates and the establishment of a TLS session.
These certify that a given public key belongs to a certain domain.
They are signed by a Certificate authority. 
The public key of the certificate authority is typically already included in your browser installation.

The binding of a public key to a specific entity (e.g. web-domain) is called 
\emph{public key infrastructure} (PKI).
\end{example}

\section{Account balances and UTXO}
The most prominent use case for blockchains is currently digital currency. 
We now spend some time to look at how to implement money transfer. 

\question{How would you implement a digital bank?}

\begin{idea}
	\label{idea-pk}
	We can use public keys as identity for users or owners of money. 
\end{idea}
\begin{note}
Idea~\ref{idea-pk} allows to avoid the use of \emph{public key infrastructure} (PKI), that is otherwise used to bind public keys to identities.
\end{note}

\textbf{Algorithm~\ref{alg:bank}} shows a simple bank that maintains balances.
For every public-key, the bank stores one balance.
The algorithm has two transactions, \textsc{Create} creates new balances.
\textsc{Transfer} allows to transfer money from one account (\textit{pk-from}) to another (\textit{pk-to}.
The algorithm assumes that balances that have not been initialized have the value 0.


\paragraph{Authentication} of transactions poses a problem:
\begin{itemize}
	\item For \textsc{Create} transactions we assume that they are are only valid during setup of the system.
	\item For \textsc{Transfer} transactions it is desired that all transactions can be submitted by users. To authenticate the sender, the transaction includes a signature $\sigma$.
	In Line~\ref{line:if} we therefore check that the sender \textit{pk-from} signed the transaction.
\end{itemize}

\noindent
A bank based on Algorithm~\ref{alg:bank} may be susceptible to \textbf{replay attacks}. A \textsc{Transfer} transaction signed by \textit{pk-from} may be submitted multiple times, given that the account of \textit{pk-from} has sufficient fonds. This will result in additional fonds transferred to \textit{pk-to}.

Later, we will see, how replay attacks in this account based model can be avoided using a counter/sequence number for each account.


\begin{algorithm}[ht!]
	\caption{Simple Bank using account balances}
	\label{alg:bank}
	\begin{algorithmic}[1]
		\State{$balances :=[pk]\textnormal{uint}$}
		\Procedure{Create}{$value, pk$}
			
				\State{$balances[pk] += value$}
			
		\EndProcedure
		\Procedure{Transfer}{$value, \textit{pk-from}, \textit{pk-to}, \sigma$}
			\If{\label{line:if} $\text{verify}(m, \mathit{pk-from}, \sigma)$} 
			\Comment{$m=value||\textit{pk-from}||\textit{pk-to}$}
 				\If{$balances[\textit{pk-from}] > value$ }
					\State{$balances[\textit{pk-from}] -= value$}
					\State{$balances[\textit{pk-to}] += value$}
				\EndIf
			\EndIf
		\EndProcedure
	\end{algorithmic}
\end{algorithm}

\question{How could you implement this bank using a blockchain?}

\begin{note}
\begin{itemize}
	\item[+] Transfer transaction can be submitted by a sender without the receiver being online.
	\item[-] It is possible to (accidentally) send money to a public key that does not exist, i.e. that nobody knows the private key for. 
\end{itemize}
\end{note}

\paragraph{Who runs the bank?}
In a centralized system, a trusted party could process transactions, compute balances and distribute balances to all parties. 

Without a trusted party, it is necessary that transactions are distributed to everybody. Every party can then deterministically compute the balances.

\emph{If all participants get all transactions, they can each process them individually and arrive at the same balances.}


\subsection{UTXO}
Bitcoin does not use balances. Instead it uses the 
\emph{unspent transaction output} (UTXO) model. We now explain this model:

\begin{idea}[UTXO]
In the UTXO model, instead of storing a balance for each private key, we store for every coin, which private key it belongs to. 
This is complicated a bit, since we want to be able to split and merge coins.

Therefore, instead of individual coins, we store unspent transaction outputs, i.e. money received and not yet spend.
\end{idea}

\begin{definition}
	A \textbf{transaction output} is a tuple $(v, pk)$ that shows, that $v$ funds have been transferred. $pk$ is a \emph{spending condition} that must be met to spend claim $v$. Typically $pk$ requires a signature with a given public key.
	
	A \textbf{transaction input} is a tuple consisting of a reference to a transaction output and an argument that meets the outputs condition.
	I.e. $(outp_i,\sigma)$ where $\sigma$ is \emph{redeeming argument} a matching $pk$, e.g. a signature.
	
	A \textbf{transaction} is a tuple containing a list of transaction inputs and a list of new outputs.
\end{definition}


\begin{note}
	\label{bitcoin:transactions}
	In bitcoin transactions are implemented in the following way:
	\begin{itemize}
		\item An output from transaction $t$ is identified by a tuple $(h_t,i)$,
		where $h_t$ is the hash of $t$ and $i$ is the index in the list of outputs in $t$.
		\item Algorithm~\ref{alg:transact} shows how a transaction is validated.
		For a transaction to be valid, \emph{all inputs must be unspent}, input \emph{signatures must validate} and the \emph{sum of input values must be larger than the  output values}.
		\item Algorithm~\ref{alg:transact} ensures that a transaction can only be validated once and no two valid transactions can spend the same output.
		\item The different between transaction inputs and outputs is called transaction \emph{fee}.
		\item Example~\ref{ex:P2PKH} gives an example for more complex conditions that may be required to spend an output.
		\item When the value of inputs is larger than the desired value to be spend, it is common to create an additional output that contains change.
	\end{itemize}
	
\end{note}

\begin{algorithm}[h!]
	\caption{Transaction validation and maintenance of UTXO}
	\label{alg:transact}
	\begin{algorithmic}
		\State{$UTXO := map[(h,i)]\rightarrow(value, pk)$}
		\Procedure{transfer}{$inputs, outputs$}\Comment{Transaction $t$ with hash $h_t$}
			\For{$((h,i),\sigma)\in inputs$}
				\If{$UTXO[(h,i)]$ does not exist}
					\State{\textbf{abort}}
					\Comment{invalid transaction}
				\EndIf
				\If{$\text{verify}(h_t,\sigma, UTXO[(h,i)].pk) == \false$}
					\State{\textbf{abort}}
					\Comment{invalid transaction}
				\EndIf
			\EndFor
			\If{ sum of values of inputs $<$ sum of values of new outputs}
				\State{\textbf{abort}}
				\Comment{invalid transaction}
			\EndIf
			
			\For{$((h,i),\sigma)\in inputs$}
				\State{$UTXO[(h,i)]=nil$} \Comment{output spent}
			\EndFor
			\State{$h_t:= hash(transaction)$}
			\State{$UTXO[h_t]=outputs$}
			\Comment{add new output}
			
			
		\EndProcedure
	\end{algorithmic}
\end{algorithm}

\begin{example}
Alice's public key has two unspent outputs with value 1\$ and 1.5\$. Alice wants to send 2\$ to Bob.
To do that, Alice can create a transaction that has her two outputs as input 
and creates two outputs, one with value 2\$ and Bobs public key. One with value 0.5\$ and Alices public key.
\end{example}


\begin{definition}
A \emph{double-spend} is a situation where multiple transactions attempt to spend the same output. 	
\end{definition}
\begin{note}
Note that according to Algorithm~\ref{alg:transact} only one of double-spend transactions can be validated. 
\end{note}

% \pagebreak


\begin{example}
	\label{ex:P2PKH}
	The following are the most common examples for arguments necessary to claim a transaction output. In Bitcoin they are expressed in a stack based scripting language.
	\begin{enumerate}[label=\alph*)]
		\item A signature that matches a certain public key.
		\item A public key that hashes to a certain value and a signature that matches this key. (Pay to public key hash or P2PKH).
		\item Multiple signatures that match a sequence of public keys. (Multisig)
		\item $m$ signatures that match $m$ out of $n$ provided public keys. 
		E.g. 2 signatures from 2 out of 3 specified public keys.
		\item A script that hashes to a certain value and an argument that causes this script to evaluate to true. (pay to script hash)
	\end{enumerate}
	See \href{https://github.com/bitcoinbook/bitcoinbook/blob/develop/ch06.asciidoc#script-construction-lock--unlock}{Mastering Bitcoin book, Chapter 6, Script Contruction}  and \href{https://d28rh4a8wq0iu5.cloudfront.net/bitcointech/readings/princeton_bitcoin_book.pdf}{Book: Bitcoin and Cryptocurrency Technologies, Chapter~3.2} for explanation of P2PKH script.
\end{example}

\begin{note}
To maintain a copy of our transaction based bank, a node has to maintain the set of all unspent transaction outputs $UTXO$. 

If variant b) is used instead of variant a) from Example~\ref{ex:P2PKH} this may significantly reduce the size of the $UTXO$ data structure. The same holds, if d) is used instead of c).
\end{note}	

\comment{On blackboard, give example of P2PKH script.}

P2PKH makes it possible to pay to the hash of a public key.
This gives rise to the concept of an address.

\begin{definition} An \textbf{address} is either a public key or the hash of a public key. Given the address of a user, it is possible to transfer funds to this user, i.e. create an output that this user can claim by providing a correct signature.
\end{definition}

\begin{note}
Bitcoin and many other cryptocurrencies use Base-58 encoding. This encoding uses small and large letters (a to z) and numbers, omitting 0 (number zero), O (capital o), l (lower L) and I (capital i) because of their ambiguity.

Bitcoin addresses use Base58Check encoding which adds a 4 byte checksum before Base-58 encoding, to protect against typos, ...
\end{note}


\subsection{Privacy in the UTXO model}
\comment{Not covered in lecture}
Different from the account and balance system, the $UTXO$ model encourages the use of different addresses. This makes it harder to identify all transactions done by a single user.

However research has shown, that based on transactions, it is easy to identify different addresses belonging to the same user.

On the other hand, $UTXO$ allows to trace in which transactions a particular value or coin was involved.

\begin{definition}
A \textbf{tainted coin} is a transaction output that is either the result of a transaction considered unethical or illegal or derived from the output of such a transaction by multiple other transactions.
\end{definition}

\begin{note}
Based on the concept of tainted coins it is debatable whether digital cash based on the UTXO model is \emph{fungible}. In economics fungibility is defined as the property that any two units of a good are interchangeable.
\end{note}

\question{What about paper money? Is it fungible?}

The UTXO model allows to create a mixing service:
\begin{definition}
	A \emph{mixing service} can be used to prevent a third party from tracking a specific users transactions. A mixing service would receive payments from many users, and pay them back using new addresses. 
\end{definition}

\begin{note}
	\begin{itemize}
		\item A mixing service makes it hard to see which of the new addresses belongs to which of the users that sent money to the mixing service.
		\item A mixing service usually requires a high fee.
		\item A mixing service is usually implemented as a centralized entity.
	\end{itemize}
\end{note}