%!TEX root = ../main.tex

\noindent
In the following we define a set of rules that correct nodes should follow. We assume that at least $c_{min}$ nodes are following these rules.

\begin{figure}
	\centering
	%!TEX root = ../main.tex

\begin{tikzpicture}
	[block/.style={rectangle,draw,thick, minimum size=1cm},
	dblock/.style={block, dashed},
	noblock/.style={block, white},
	prev/.style={draw, -latex, thick}, 
	dprev/.style={prev, dashed}, 
	interval/.style={|-|, dashed, thick},
	node distance =.5cm and 1cm]
	\node(b0)[block]{$b_0$};
	\node(b1)[block, right=of b0]{$b_1$};
	\draw[prev](b1) -- (b0);
	\node(b2)[block, right=of b1]{$b_2$};
	\draw[prev](b2) -- (b1);
	\node(b3)[noblock, right=of b2]{};
	\node(b3a)[block, above right=of b2.east]{$b_3'$};
	\draw[prev](b3a) -- (b2);

	\node(b3b)[dblock, below right=of b2.east]{$b_3''$};
	\draw[dprev](b3b) -- (b2);
	
	\node(b4)[noblock, right=of b3]{};
	\node(b4b)[dblock, right=of b3b.east]{$b_4''$};
	\draw[dprev](b4b) -- (b3b);
	\node(b4a)[block, right=of b3a.east]{$b_4'$};
	\draw[prev](b4a) -- (b3a);

	\node(b5)[noblock, right=of b4]{};
	\node(b5b)[dblock, right=of b4b.east]{$b_5''$};
	\draw[dprev](b5b) -- (b4b);
	
	
	
	\foreach \x in {0, 1, 2,3,4}{
	\draw[interval] ([xshift=-.4cm,yshift=-2cm]b\x.west) -- node[below]{$\depth=\x$} ([xshift=.4cm,yshift=-2cm]b\x.east);};
	
\end{tikzpicture}
\caption{Blocks in a tree and depth $d$ of blocks.}
\label{fig:tree}
\end{figure}

The \emph{previousBlock} pointers create a tree structure on blocks. We can thus refer to the \emph{depth} of a block. We write $b.\depth$ for the depth of block $b$. Figure~\ref{fig:tree} shows a tree with different depth levels. Here $b_4''.\depth=4$. Similarly, we refer to the parent, ancestors or descendants of a block. In Figure~\ref{fig:tree}, e.g., $b_5''$ is a descendant of $b_3''$ and all blocks are descendants of $b_0$. Further, $b_3'$ is the parent of $b_4'$ and $b_2$ is an ancestor of $b_4'$. 

We can not define the first rule.
\begin{Rule}
After signing a block at depth $d$, only sign at depth $d'>d$.
\end{Rule}
\noindent
The first rule says that nodes should only sign at increasing depth. Especially, they should not sign at the same depth twice.

As noted in Section~\ref{sec:poc}, to prevent forks we should also add a rule that prohibits changing from one branch of the tree to another, e.g., signing  $b_5''$ in Figure~\ref{fig:tree} after signing $b_3'$ and $b_4'$.
%
We further note, that it is possible, that no block at a certain depth receives a certificate. E.g.,~$b_3'$ and $b_3''$ may both be signed by 5 out of 10 nodes. Thus, no block has the required ($c_{min}(10)=7$) signatures.  
%
Therefore it should still be allowed for a node to sign $b_4'$ after signing $b_3''$. 
%

\begin{definition}The \emph{locked block} at node $n_i$, $n_i.\lock$ is the block $b$ at highest depth, such that $n_i$ has (or has seen) a certificate for $b$.
\end{definition}

\begin{Rule} 
	A node $n_i$ only signs a block that is a descendant of $n_i.\lock$.	
\end{Rule}

We note that if $c_{min}$ nodes sign a block $b$, that does not imply that these nodes know the certificate of this block. Certificates have to be collected and disseminated by some node. We therefore define the following:

\begin{definition} We assume that every block $b_i$ contains a certificate for a block $b_j$ such that $b_j$ is an ancestor of $b_i$. We say that $b_i.\just=b_j$. 	
\end{definition}

\begin{lem}
	For some node $n_i$ that follows Rule 2, it holds that, after signing block $b$, $n_i.\lock.\depth\geq b.\just.\depth$ holds.
\end{lem}

\begin{example} In Figure~\ref{fig:tree}, if $b_4'.\just=b_3'$, then after signing $b_4'$ a node following Rule~2 will not sign $b_5''$.	
\end{example}

\begin{figure}
	\centering
	%!TEX root = ../main.tex

\begin{tikzpicture}
	[block/.style={rectangle,draw,thick, minimum size=1cm},
	dblock/.style={block, dashed},
	noblock/.style={block, white},
	prev/.style={draw, -latex, thick}, 
	dprev/.style={prev, dashed},
	just/.style={-latex, double, thick, red},
	justarc/.style={out=135, in=45},
	interval/.style={|-|, dashed, thick},
	node distance =.5cm and 1cm]
	\node(b0)[block]{$b$};
	\node(b1)[block, right=of b0]{$b'$};
	\draw[prev](b1) -- (b0);
	\node(b2)[block, right=of b1]{$b''$};
	\draw[prev](b2) -- (b1);
	
	
	\foreach \from/\to in {b1/b0, b2/b1}{
		\draw[just] (\from) to[justarc] (\to);
	}
	
\end{tikzpicture}
	\caption{Blockes with red justification link, confirm block $b$.}
\end{figure}

\begin{definition}
\label{def:confirmed}	
We say that a block $b$ is \emph{confirmed}, if there exist blocks $b'$ and $b''$, such that, $b=b'.\just$, $b'=b''.\just$, and $b.\depth = b''.\depth-2$.	
\end{definition}

We note that the notion of a confirmed block is similar to proof of work blockchains like bitcoin, where a block counts as confirmed if it has been extended by a certain number of blocks, e.g. 6 blocks in bitcoin.

\begin{theorem}\label{thm:confirmed}
If nodes follow Rule~2 and a block $b$ is confirmed, then any certified block at depth $d>b.\depth$ is a descendant of $b$.
\end{theorem}

\begin{proof}
If $b$ is confirmed, there exist $b'$ and $b''$ as in Definition~\ref{def:confirmed}. $b''$ contains a certificate for $b'$. Thus at least $c_{min}$ nodes have signed $b'$. $b'$ contains a certificate for $b$. Thus upon signing $b'$, a node $n_i$ sets $n_i.\lock=b$.

We have to show that every certified block $\beta$ with $\beta.\depth > b.\depth$ is a descendant of $b$. We proof this by induction over $\Delta_d=\beta.\depth-b.\depth$. 

If $\Delta_d=1$ then $\beta=b'$ since at most one block at depth $b.\depth+1$ can be certified. 

Let $\Delta_d=n+1$. Both $\beta$ and $b'$ are signed by $c_{min}$ nodes. 
Let $n_i$ be a correct nodes (following Rules~1 and~2) that signed both $\beta$ and $b'$. Due to Rule~1,  $n_i$ signed $b'$ before $\beta$. 
Thus $n_i.\lock=b$ did hold. The induction hypothesis implies that when signing $\beta$, $n_i.\lock$ was either $b$ or a descendant of $b$. Rule~2 implies that $\beta$ is a descendant of $n_i.\lock$. Thus $\beta$ is a descendant of $b$.
\end{proof}